\documentclass[11pt,a4paper]{article}
\usepackage[utf8]{inputenc}
\usepackage[T1]{fontenc}
\usepackage{geometry}
\usepackage{hyperref}
\usepackage{amsmath}
\usepackage{titlesec}
\usepackage{enumitem}
\usepackage{xcolor}

\geometry{margin=1in}

% Custom section formatting
\titleformat{\section}{\large\bfseries\uppercase}{}{0pt}{}[\titlerule]
\titlespacing*{\section}{0pt}{12pt}{8pt}

\begin{document}

\begin{flushleft}
    \textbf{\huge EXECUTIVE SUMMARY: Project CODE-GEO (V3.1)} \\
    \vspace{2mm}
    \textbf{Subject:} Resolution of the GW250114 Ringdown Anomaly via Complexity-Gated Horizon Shells \\
    \textbf{Date:} February 21, 2026 \\
    \textbf{Lead Researcher:} Shane Hartley \\
    \textbf{Contact:} darian.frey@yahoo.com
\end{flushleft}

\vspace{5mm}

\section{1. The Core Problem}
The O4b event \textbf{GW250114} ($M_{rem} \approx 62.7 M_{\odot}$) exhibits a secondary spectral residue at \textbf{355 Hz} that is not fully accounted for by standard Kerr quasi-normal mode (QNM) overtones. Traditional General Relativity provides no mechanism for horizon-scale reflections without violating the ``no-hair'' theorem.

\section{2. The CODE-GEO Solution}
We propose a modified field theory where spacetime curvature is coupled to \textbf{Quantum Complexity Density}.

\begin{itemize}[leftmargin=*, labelsep=10pt]
    \item \textbf{Hilbert-Complexity Action:} Introduces a term $\alpha \mathcal{C}_K$, where $\alpha$ is Planckian-scale but is promoted to macroscopic significance at the horizon through \textbf{Large-N holographic scaling}.
    \item \textbf{The Nonlinear Gate:} To satisfy high-precision inspiral data, the theory utilizes a $(R_s/r)^6$ suppression. This keeps the theory ``Stealth'' until the final merger, preventing any conflict with established LIGO/Virgo/KAGRA (LVK) results for binary pulsars or early-stage inspirals.
    \item \textbf{The Fuzzy Shell:} At the moment of merger, a refractive shell inflates to \textbf{2.0 $R_s$} ($\sim$371 km). This shell acts as a ``Slow-Light'' medium (refractive index $n \approx 4.56$).
\end{itemize}

\section{3. Predicted Observables}
\begin{itemize}[leftmargin=*, labelsep=10pt]
    \item \textbf{Echo Delay:} Exactly \textbf{2.816 ms} post-merger.
    \item \textbf{Spectral Fingerprint:} A 70.5 Hz resonance gap between the Kerr fundamental (284.6 Hz) and the CODE-GEO complexity echo (355.11 Hz).
    \item \textbf{Information Paradox:} Resolves the paradox by ensuring information is scrambled and reflected at the $2.0 R_s$ surface rather than lost to a singularity.
\end{itemize}

\section{4. Immediate Research Action}
All simulation scripts and the full technical derivation (V4.0) are available for audit at: \\
\href{https://github.com/Darian-Frey/CODE-GEO}{\texttt{https://github.com/Darian-Frey/CODE-GEO}}

We invite the LSC Data Analysis Council to apply our \textbf{matched-filter parameters} to the raw strain data of GW250114 to verify the existence of the 2.816 ms echo.

\end{document}
